\documentclass[11pt,a4paper,french,twoside]{VcCours}
\begin{document}
\section*{Exercice ? (4 points)}
\emph{Cet exercice traite principalement le thème \og{} algorithmique
langages et programmation\fg{}.}

L'objectif de cet exercice est de permettre, dans le service des urgences
d'un hôpital, la prise en compte de la gravité des cas dans la gestion
de l'ordre de passage des patients.

\medskip
On choisit de représenter la file d'attente par un tableau (type \code{list})
dont les éléments sont de objets définis par la classe suivante :
\begin{Python}
class Patient:
    def __init__(self, nom, prenom, age, gravite):
        self.nom=nom
        self.prenom=prenom
        self.age=age
        self.gravite=gravite
\end{Python}
Chaque patient est donc défini par son nom, son prénom, son âge (en années)
et la gravité de son cas (qui est un entier de $1$ à $10$, $1$ correspondant
aux cas sans urgence et 10 aux cas d'urgence vitale).

\begin{enumerate}
    \item On exécute une à une les instructions suivantes :
\begin{Python}
attente=[]
attente.append(Patient("Martin","Abel",35,1))
attente.append(Patient("Dupond","Béatrice",42,5))
attente.append(Patient("Laurent","Charles",60,2))
\end{Python}
\begin{enumerate}
    \item Quel est le résultat de l'exécution de \code{L[1].age} ?
    \item Quel est le résultat de l'exécution de \code{L[2].prenom[1]} ?
    \item Quelle commande exécuter pour passer la gravité de Béatrice à $7$ ?
\end{enumerate}    
    \item On considère la fonction suivante:
\begin{Python}
def mystere(attente):
    A=0
    for k in range(len(attente)):
        if attente[k].age>A:
            A=attente[k].age
    return A
\end{Python}
    Pour une liste \code{attente} d'objets de classe \code{Patient} quelconque,
    que fait cette fonction ?
    \item Écrire (en s'inspirant de la question précédente) une fonction \code{cas\_critique(attente)} qui détermine
    le cas le plus grave ou, s'il y a égalité, le plus jeune.
    Cette fonction doit retourner le rang du patient en question dans la liste 
    \code{attente}.
    \item On exécute une à une les instructions suivantes :
\begin{Python}
attente=[]
attente.append(Patient("Martin","Abel",35,1))
attente.append(Patient("Dupond","Béatrice",42,5))
attente[0]=attente[1]
attente[1]=attente[0]
\end{Python}
    Choisissez une réponse parmi les cinq suivantes :
    \begin{enumerate}
        \item La liste \code{attente} contient deux fois le patient Martin. 
        \item La liste \code{attente} contient deux fois la patiente Dupond. 
        \item La liste \code{attente} est inchangée. 
        \item Les patients Martin et Dupond ont échangé leur place dans la liste \code{attente}. 
        \item Une erreur se produit à l'exécution. 
    \end{enumerate}
    \item On exécute une à une les instructions suivantes :
\begin{Python}
attente=[]
attente.append(Patient("Martin","Abel",35,1))
attente.append(Patient("Dupond","Béatrice",42,5))
x=attente[1]
attente[1]=attente[0]
attente[0]=x
\end{Python}
    Choisissez une réponse parmi les cinq suivantes :
    \begin{enumerate}
        \item La liste \code{attente} contient deux fois le patient Martin. 
        \item La liste \code{attente} contient deux fois la patiente Dupond. 
        \item La liste \code{attente} est inchangée. 
        \item Les patients Martin et Dupond ont échangé leur place dans la liste \code{attente}. 
        \item Une erreur se produit à l'exécution. 
    \end{enumerate}
    \item Écrire une fonction \code{doubler(attente,k)}, qui compare les patients
    correspondants aux index $k-1$ et $k$ et les échange éventuellement pour que
    le cas le plus grave (le plus âgé, s'ils ont la même gravité) soit en position $k-1$.
    \item Écrire une fonction \code{suivant(attente)}, qui appliqué la fontion 
    \code{doubler} à toute la file d'attente, en commençant par les plus petits 
    index. Puis renvoie le premier patient en le retirant la liste.
    \item On souhaite modifier la fonction \code{doubler(attente,k)} pour qu'elle
    déplace le patient d'index $k$ vers la gauche et ce tant que le total de 
    gravité des patients qu'il dépasse est strictement inférieur 
    à sa propre gravité. (Un patient de gravité $5$ pourra dépasser les trois 
    patients qui le précèdent si leurs gravités sont $1$, $2$, $1$, 
    mais il s'arrêtera là car $1+2+1=4<5$.)
    
    Compléter le code suivant :
\begin{Python}
def doubler(attente,k):
    g=attente[k].gravite
    i=k-1
    S=0
    while ____ and _______________:
        S=S+attente[i].gravite
        attente[i],attente[i+1]=_________________________
        i=___
\end{Python}
\end{enumerate}

\end{document}