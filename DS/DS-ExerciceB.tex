\documentclass[11pt,a4paper,french,twoside]{VcCours}
\begin{document}
\section*{Exercice ? (4 points)}
\emph{Cet exercice traite principalement le thème \og{} algorithmique
langages et programmation\fg{}.}

L'objectif de cet exercice est de définir un moyen de de définir l'ordre 
d'exécution des processus dans un système d'exploitation multitâche sur un 
ordinateur ayant un unique processeur. En effet, le système doit assurer
l'exécution d'un nombre important de tâches, mais cele-ci ne peuvent être
assignées au processeur qu'une par une. De plus, les tâches ont des priorités
différentes. Dans un système d'exploitation, c'est le trvail de l'ordonnanceur.

\medskip
On choisit de représenter la file d'attente des processus à exécuter par 
un tableau (type \code{list}) dont les éléments sont de objets définis 
par la classe suivante :
\begin{Python}
class Processus:
    def __init__(self, nom, chemin, duree, priorite):
        self.nom=nom
        self.chemin=chemin
        self.duree=duree
        self.priorite=priorite
\end{Python}
Chaque processus est donc défini par son nom, son chemin, sa durée (en cycles 
d'horloge) et le niveau priorité (qui est un entier de $1$ à $10$, 
$1$ correspondant à une très faible priorité et $10$ à une tâche ultra prioritaire,
autrement dit : une tâche temps réel). 

\begin{enumerate}
    \item On exécute une à une les instructions suivantes :
\begin{Python}
attente=[]
attente.append(Processus("zip.exe","C:/bin",100,1))
attente.append(Processus("chrome.exe","C:/bin",50,6))
attente.append(Processus("skype.exe","C:/users/eve",60,9))
\end{Python}
\begin{enumerate}
    \item Quel est le résultat de l'exécution de \code{L[2].duree} ?
    \item Quel est le résultat de l'exécution de \code{L[1].chemin[2]} ?
    \item Quelle commande exécuter pour passer la priorité de skype à $10$ ?
\end{enumerate}    
    \item On considère la fonction suivante:
\begin{Python}
def mystere(attente):
    D=0
    for k in range(len(attente)):
        if attente[k].duree<D:
            D=attente[k].duree
    return D
\end{Python}
    Pour une liste \code{attente} d'objets de classe \code{Processus} quelconque,
    que fait cette fonction ?
    \item Écrire (en s'inspirant de la question précédente) une fonction 
    \code{tache\_critique(attente)} qui détermine
    le processus le plus prioritaire ou, s'il y a égalité, celui qui dure le 
    plus longtemps.
    Cette fonction doit retourner le rang du processus en question dans 
    la liste \code{attente}.
    \item On exécute une à une les instructions suivantes :
\begin{Python}
attente=[]
attente.append(Processus("zip.exe","C:/bin",100,1))
attente.append(Processus("chrome.exe","C:/bin",50,6))
attente[0]=attente[1]
attente[1]=attente[0]
\end{Python}
    Choisissez une réponse parmi les cinq suivantes :
    \begin{enumerate}
        \item La liste \code{attente} contient deux fois le processus zip.exe. 
        \item La liste \code{attente} contient deux fois le processus chrome.exe. 
        \item La liste \code{attente} est inchangée. 
        \item Les processus zip.exe et chrome.exe ont échangé leur place dans 
        la liste \code{attente}. 
        \item Une erreur se produit à l'exécution. 
    \end{enumerate}
    \item On exécute une à une les instructions suivantes :
\begin{Python}
attente=[]
attente.append(Processus("zip.exe","C:/bin",100,1))
attente.append(Processus("chrome.exe","C:/bin",50,6))
x=attente[1]
attente[1]=attente[0]
attente[0]=x
\end{Python}
    Choisissez une réponse parmi les cinq suivantes :
    \begin{enumerate}
        \item La liste \code{attente} contient deux fois le processus zip.exe. 
        \item La liste \code{attente} contient deux fois le processus chrome.exe. 
        \item La liste \code{attente} est inchangée. 
        \item Les processus zip.exe et chrome.exe ont échangé leur place dans 
        la liste \code{attente}. 
        \item Une erreur se produit à l'exécution. 
    \end{enumerate}
    \item Écrire une fonction \code{depasser(attente,k)}, qui compare les processus
    correspondants aux index $k-1$ et $k$ et les échange éventuellement pour que
    le prcessus le plus prioritaire (le plus court, s'ils ont la même priorité) 
    soit en position $k-1$.
    \item Écrire une fonction \code{suivant(attente)}, qui appliqué la fontion 
    \code{depasser} à toute la file d'attente, en commençant par les plus petits 
    index. Puis renvoie le premier processus en le retirant la liste.
    \item On souhaite modifier la fonction \code{depasser(attente,k)} pour qu'elle
    déplace le processus d'index $k$ vers la gauche et ce tant que le total des 
    priorités des processus qu'il dépasse est strictement inférieur 
    à son propre niveau de priorité. (Un processus de priorité $8$ pourra 
    dépasser les trois processus qui le précèdent si leurs priorités sont $1$, $4$, $2$, 
    mais il s'arrêtera là car $1+4+2=7<8$.)
    
    Compléter le code suivant :
\begin{Python}
def depasser(attente,k):
    p=attente[k].priorite
    i=k-1
    S=0
    while ____ and ________________:
        S=S+attente[i].priorite
        attente[i],attente[i+1]=_________________________
        i=___
\end{Python}
\end{enumerate}

\end{document}