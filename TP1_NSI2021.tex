\documentclass[11pt,a4paper,french,twoside]{PMCours}
\usepackage{hyperref}

\newcounter{activite}
\newcommand{\activite}{\subsubsection*{Activité~\refstepcounter{activite}\theactivite}}

\begin{document}
\TitreISN{Classe de Terminale}{Année 2021--2022}
{Numérique et Sciences Informatiques}{TP 1 : Sommation, recherche, maximum.}

Il s'agit ici d'un TP d'application du chapitre 1 regroupant un certain nombre d'algorithmes de base.

\tableofcontents


\newpage

\section{Sommation}
\subsection{Algorithme de sommation}
On a un tableau \code{L}, on veut calculer la somme des termes du tableau.

L'idée principale est d'avoir une variable \code{S} (l'accumulateur) contenant la
somme des termes déjà lus.

Au départ \code{S} vaut \code{0}, puis on rajoute chacun des éléments du tableau à cet
accumulateur \code{S}. Quand on a fini de passer en revue tous les éléments du
tableau, \code{S} contient le total souhaité.

\begin{Python}
def somme(L) :
    """Calcule la somme des termes du tableau L."""
    n = len(L) # nombre de termes du tableau
    S = 0
    for k in range(n) :
        S = S + L[k]
    return S
\end{Python}

Un exemple d'exécution :
\begin{Python*}
somme([5,9,-2,5,8])
\end{Python*}

Le tableau suivant retrace l'exécution de la fonction \code{somme}.

{\large \begin{tabular} {|l|p{1cm}|p{1cm}|p{1cm}|p{1cm}|p{1cm}|p{1cm}|}\hline
\verb|initialisation |&\multicolumn{6}{l|}{}\\ \hline
\mbox{boucle for}&\verb|k|& & & && \\ \hline
&\verb|L[k]|& & && & \\ \hline
&\verb|S|& & && & \\ \hline
\mbox{fin boucle for}&\multicolumn{6}{l|}{}\\ \hline
 \end{tabular}}
 
\medskip
\subsection{Applications}
\activite
Écrire une fonction \code{serie(n)} qui, pour \code{n}$\in\N$, donne
$\sum_{k=1}^n\frac{1}{k^2}=\frac{1}{1^2}+\frac{1}{2^2}+\frac{1}{3^2}+\ldots
+\frac{1}{n^2}$.
	
\activite
Écrire une fonction \code{produit(L)} qui, pour un tableau \code{L}, donne
le produit de tous ses éléments.
	
\activite
Écrire une fonction \code{sommeTermesPositifs(L)} qui, pour un
tableau \code{L}, donne le somme des éléments positifs du tableau.
	
\activite
Écrire une fonction \code{moyenne(L)} qui, pour un tableau \code{L} de tuples 
\code{(valeur,coefficient)}, donne le moyenne pondérée.

Testez avec \code{moyenne([(10,4),(7,2),(15,3))}, qui doit donner $11$.

\activite
Écrire une fonction \code{somme\_double(T)} qui, pour un tableau de tableaux \code{T}, 
donne le somme des éléments du tableau.

Testez avec \code{somme\_double([[1,2,3],[4,5]])}, qui doit donner $15$.


\section{Recherche d'un élément}
\subsection{Algorithme de recherche}
On a un tableau \code{L} et une valeur \code{v}, on veut déterminer si la valeur
\code{v} est présente dans le tableau \code{L}.

(En python, cela correspond à l'expression \codePY{v in L}.)

L'idée principale est de comparer successivement chacun des termes du tableau
avec la valeur recherché et de répondre \code{True} dès qu'on la trouve. Et si 
jamais on atteint la fin du tableau sans jamais avoir trouvé la valeur, alors 
on répond \code{False}.

\begin{Python}
def recherche(L,v) :
    """Répond True si v est dans L et False sinon."""
    n = len(L) # nombre de termes du tableau
    for k in range(n) :
        if v==L[k] :
            return True
    return False
\end{Python}

Un exemple d'exécution :
\begin{Python*}
recherche([5,2,9,-2,9,5],9)
\end{Python*}

Le tableau suivant retrace l'exécution de la fonction \code{recherche}.

Il donne l'état des variables après l'exécution de chacune des lignes (donnée
par leur numéro).

\begin{center}
\begin{tabular}{|c|c|c|c|c|}
\hline
  N° ligne&\makebox[2cm]{v}&\makebox[2cm]{k}&\makebox[2cm]{n}&\makebox[5cm]{L}\\
  \hline
  1 &&&&\\
  \hline
  3 &&&&\\
  \hline
  4 &&&&\\
  \hline
  5 &&&\multicolumn{2}{l|}{Le test \code{v==L[k]} est}\\
  \hline
  4 &&&&\\
  \hline
  5 &&&\multicolumn{2}{l|}{Le test \code{v==L[k]} est}\\
  \hline
  4 &&&&\\
  \hline
  5 &&&\multicolumn{2}{l|}{Le test \code{v==L[k]} est}\\
  \hline
  6 &\multicolumn{4}{c|}{}\\
  \hline
\end{tabular}
\end{center}

\pagebreak
deuxième exemple d'exécution :
\begin{Python*}
recherche([5,-2],9)
\end{Python*}

\begin{center}
\begin{tabular}{|c|c|c|c|c|}
\hline
  N° ligne&\makebox[2cm]{v}&\makebox[2cm]{k}&\makebox[2cm]{n}&\makebox[5cm]{L}\\
  \hline
  1 &&&&\\
  \hline
  3 &&&&\\
  \hline
  4 &&&&\\
  \hline
  5 &&&\multicolumn{2}{l|}{Le test \code{v==L[k]} est}\\
  \hline
  4 &&&&\\
  \hline
  5 &&&\multicolumn{2}{l|}{Le test \code{v==L[k]} est}\\
  \hline
  4 &&&\multicolumn{2}{l|}{}\\
  \hline
  7 &\multicolumn{4}{c|}{}\\
  \hline
\end{tabular}
\end{center}

\subsection{Applications}
% TODO: Applications à revoir.
\activite
Écrire une fonction \code{recherche2} qui prend comme argument un tableau 
\code{L} et un élément \code{a} et renvoie la position de l'élément s'il est 
présent, ou \code{-1} sinon.

Par exemple, avec \code{L=[2,7,-8,2,2,9]},\\
\code{recherche2(L,2)} donne \code{0} ou \code{3} ou \code{4} (fonction de votre
programmation) et \code{recherche2(L,15)=-1}

\activite
Écrire une fonction \code{recherche3} qui prend comme argument un tableau 
\code{L} et un élément \code{a} et renvoie un tableau \code{T} contenant les 
indices des occurrences de \code{a} dans \code{L}.

Par exemple, avec \code{L=[2,7,-8,2,2,9]},\\
\code{recherche3(L,2)=[0,3,4]} et \code{recherche3(L,15)=[]}

\activite
Écrire une fonction Python \code{positions(sequence,codon)}, où \code{sequence}
et \code{codon} sont deux chaînes de caractères, qui retourne le
tableau des positions du \code{codon} dans \code{sequence}. Pour être plus précis, 
la position du codon est l'index de sa première lettre.

Par exemple, avec \code{seq="ACGTCACCGTCA"},\\
\code{positions(seq,"GTC")=[2,8]} et \code{positions(seq,"ACCTGA")=[]}


\newpage
\section{Recherche du maximum}
\subsection{Algorithme du maximum}
On a un tableau \code{L}, on veut déterminer le plus grand des termes
du tableau.

L'idée principale est d'avoir une variable \code{M} (la mémoire)
contenant le plus grand terme rencontré jusque là.

Au départ \code{M} prend la valeur du premier terme du tableau, puis on compare
chacun des autres termes à \code{M} et, s'il est plus grand, on met à jour 
\code{M} avec cette valeur.
Quand on a fini de passer en revue tous les éléments du tableau, \code{M} 
contient le plus grand de tous.

\begin{Python}
def maximum(L) :
    """Donne le plus grand des termes du tableau L et l'indice auquel on le rencontre."""
    n = len(L) # nombre de termes du tableau
    M = L[0]
    i=0
    for k in range(1,n) :
        if L[k]>M :
            M = L[k]
            i=k
    return M,i
\end{Python}

Un exemple d'exécution :
\begin{Python*}
maximum([5,2,9,-2,11,5,7])
\end{Python*}

Le tableau suivant retrace l'exécution de la fonction \code{maximum}.

\vspace{1cm}
{\large \begin{tabular} {|l|p{1.3cm}|p{1.3cm}|p{1.3cm}|p{1.3cm}|p{1.3cm}|p{1.3cm}|p{1.3cm}|}\hline
\verb|initialisation |&\multicolumn{7}{l|}{}\\ \hline
\mbox{boucle for}&\verb|k|&&&&&&\\ \hline
&\verb|L[k]|& & && & &\\ \hline
&\verb|Test|& & && & &\\ \hline
&\verb|M|& & && & &\\ \hline
&\verb|i|& & && & &\\ \hline
\mbox{fin boucle for}&\multicolumn{7}{l|}{}\\ \hline
\end{tabular}}
 
\medskip
\subsection{Applications}
\activite
Écrire une fonction \code{position\_du\_max(L)} qui, pour un tableau \code{L}, 
donne la/une position du plus grand des éléments du tableau.

\activite
Écrire une fonction \code{positions\_du\_max(L)} qui, pour un tableau \code{L},
donne le tableau des positions du plus grand des éléments du tableau.

\activite
Écrire une fonction \code{maximum\_double(T)} qui, pour un tableau de tableaux 
\code{T}, donne le plus grand de tous les éléments.
\section{Tableau en compréhension}
\subsection{Généralités}
Nous avons vu dans le cours comment définir un tableau {\em par énumération}, 
c'est à dire en énumérant chacun des éléments du tableau.

Une autre méthode existe dans certains langages informatiques.

\begin{Definition}{}
    On dit qu'un tableau est défini en compréhension si ses 
    éléments sont le résultat d'opérations (dont le filtrage) des éléments 
    d'un autre tableau.
\end{Definition}

Par exemple, 

\begin{Python}
t=[2*k for k in range(1,5)]
\end{Python}

donne \code{t=[2,4,6,8]}

\begin{Python}
t=[2,-5,-7,8,9,-9,10,3]
p=[k for k in L if k>0]
\end{Python}

donne \code{p=[2,8,9,10,3]}

\begin{Python}
t=[(i,j,i-j) for i in range(1,5) for j in range(1,5) if i<j]
\end{Python}

donne \code{t=[(1, 2, -1), (1, 3, -2), (1, 4, -3), (2, 3, -1), (2, 4, -2), (3, 4, -1)]}.

C'est une manière de procéder extrêmement efficace mais qui ne doit pas 
faire oublier la pratique classique de l'algorithmique.

\subsection{Applications}
\activite
En utilisant la définition en compréhension des tableaux, 
écrire une commande donnant tous les codes PIN constitués de 4 chiffres croissants.

\activite
Déterminer tous les triples d'entiers positifs $a$, $b$ et $c$ satisfaisant à 
$\left\{\begin{array}{l}1\leq a<b<c\leq 1000 \\a^2+b^2=c^2\end{array}\right.$

(Ce sont des \emph{triplets Pythagoricien}).

\section{Utilisation de Dictionnaires}
\activite 
{\bf Dépouillement d'un scrutin :}
On dispose d'une liste contenant des noms correspondants au résultat d'un vote 
et on veut automatiser le comptage des voix obtenues par chacun.

Un moyen basique serait de passer par une série de test et d'incrémenter 
des compteurs. Mais on peut peut être plus astucieux\dots

\begin{enumerate}
\item Écrire une fonction \code{depouille} prenant comme argument 
une liste \code{l} correspondant aux noms ayant recueilli les suffrages 
et renvoyant un dictionnaire comptabilisant les suffrages obtenus par chacun.

Par exemple, avec \code{l=['Nadia','Eva','Eva','Jean','Nadia','Eva']}, 
le résultat renvoyé sera \code{\{'Nadia':2,'Jean':1,'Eva':3\}}.

\item On va maintenant, au moyen des fonctions de tri intégrées à Python, 
trier ce dictionnaire et en déduire le vainqueur de l'élection.
\begin{enumerate}
\item Au moyen de la méthode \code{.items()}, 
créer une liste \code{l\_elec} contenant, sous forme de 2-uplets, 
la liste des items du dictionnaire.
\item Exécuter en console la commande \code{sorted(l\_elec)}. 

Suivant quel critère \code{l\_elec} a-t-elle été triée ?
\item On a besoin d'une {\em clef de tri} permettant d'indiquer qu'il faut trier
les éléments de \code{l\_elec} suivant leur deuxième composante.

Écrire le script d'une fonction \code{clef\_de\_tri} prenant comme argument 
un 2-uplet \code{a} et renvoyant sa seconde composante.

Exécuter en console la commande \code{sorted(l\_elec,key=clef\_de\_tri)} 
puis la commande \code{sorted(l\_elec,key=clef\_de\_tri,reverse=True)}. 

En déduire le vainqueur de l'élection.

\end{enumerate}
\end{enumerate}

\activite 
\textbf{Cryptage/Décryptage} : 
un cryptage mono-alphabétique, comme le code de césar, 
peut être résumé par un simple dictionnaire 
dont les clefs sont les lettres de l'alphabet dans l'ordre classique 
et les valeurs sont les lettres correspondantes dans le cryptage.

Par exemple, si notre code transforme \code{''A''} en \code{''N''}, 
alors le dictionnaire contiendra le couple clef-valeur \code{''A'':''N''}.

\begin{enumerate}
\item Taper les commandes suivantes :
% cSpell:disable
\begin{Python}
alphabet="ABCDEFGHIJKLMNOPQRSTUVWXYZ "
alphabet_code=" NBVCXWQSDFGHJKLMPOIUYTREZA"
dico_codage=dict(zip(alphabet,alphabet_code))
\end{Python}
% cSpell:enable
Notez que la commande \code{zip} peut aussi être utilisée 
avec des types \code{list},\dots
\item Écrire une fonction \code{codage\_mono} prenant comme argument 
un texte écrit en majuscules (sans accents, ponctuation,\dots{} mais avec espaces) 
et renvoyant le texte codé.

Écrire une fonction \code{decodage\_mono} permettant d'effectuer 
l'opération inverse.
\end{enumerate}

\end{document}
