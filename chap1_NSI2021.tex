\documentclass[11pt,a4paper,french,twoside]{PMCours}
\usepackage{hyperref}
% TODO: Ajuster la position des sauts de page.
\begin{document}
\TitreISN{Classe de Terminale}{Année 2021--2022}
{Numérique et Sciences Informatiques}{Chap. 1 : Rappels}
\section{Généralités sur les variables}
\subsection{Définition}
Le premier élément de base de la programmation est la \emph{variable}, qui permet
de stocker les données utilisées par le programme.

Pour simplifier,

\begin{Definition}{}
Une variable est l'association d'un nom (l'identifiant) et d'une valeur via 
l'indication d'un emplacement mémoire.

Les variables ont un type, qui définit les modalités de stockage en 
mémoire, la nature des valeurs qu'elles peuvent prendre et les opérations que 
l'on peut exécuter avec.

Donner une valeur à une variable s'appelle réaliser une affectation. 
\end{Definition}

\medskip
Diverses remarques : 
\begin{itemize}
	\item En langage Python, les identifiants des variables peuvent contenir des 
	chiffres et des lettres (mais ne peuvent commencer par un chiffre) et le 
	symboles $\_$ (underscore : tiret bas, touche 8). Ils sont \emph{case-sensitive} 
	c'est à dire sensible aux majuscules/minuscules.
	\item Certains langages, comme le C, sont à \emph{typage statique} : les 
	variables et leurs types doivent être déclarées préalablement et ne peuvent 
	accueillir que des valeurs de ce type.

	D'autres langages, comme Python, sont à \emph{typage dynamique} et les 
	variables reçoivent leurs types lors de l'affectation et ce type peut 
	changer au cours de l'exécution d'un programme.  
\end{itemize}

\subsection{Les différents types simples}
En Python, on accède au type d'une variable via la commande \code{type}. On 
distingue les types simples suivants :
\begin{itemize}
\item \code{int} : entiers signés, sans limitation de taille (\emph{entiers longs}).
\item \code{float} : nombres à virgule flottante codés sur 64 bits (mantisse : 52 bits,
exposant : 11 bits, signe : 1 bit) et allant environ de $5\cdot10^{-324}$ à $2\cdot 10^{308}$.
\item \code{str} : chaînes de caractères, délimitées par des guillemets simples 
ou doubles.
\item \code{bool} : booléens, ce sont des variables pouvant prendre comme valeur 
\code{True} et \code{False}, correspondant aux propositions logiques. 
\end{itemize}
\subsubsection*{Activité 1}
Dans la console Python, tapez les commandes suivantes et commentez les résultats :
\begin{Python}
a=5
b=1.23
type(a)
type(b)
a+b
type(a+b)
c=a>b
type(c)
d='coucou'
type(d)
'5'+'5'
5+5
5+'5'
int(7.5)
int('75')
str(71.23)
\end{Python}
On reverra dans la suite du chapitre les types construits de Python : tableau, 
tuple, dictionnaire. 
\subsection{Les règles de l'affectation}
Les emplacements mémoires où peuvent être stockées les variables Python sont 
repérés par des entiers auxquels ont peut accéder par la commande \code{id}.

En langage Python, l'affectation d'une variable se fait par une écriture du type

\smallskip
\centerline{\code{identifiant de variable = expression}}

\smallskip
et obéit aux règles suivantes :
\begin{itemize}
\item si l'expression est réduite à un nom de variable, la nouvelle variable se
 voit attribuer le même emplacement mémoire / le même \code{id} que la variable 
 de droite. Ainsi, les deux noms de variables sont juste deux accès différents 
 vers une même ressource. %c'est l'aliasing : à dire à l'oral.
\item sinon, l'expression est évaluée et :
\begin{itemize}
% On dira : immuable ou non mutable. 
% zapper le vocabulaire sur l'interning, trop spécialisé.
\item si elle est d'un type simple et que la valeur a déjà été stockée en 
mémoire, alors \uline{sauf exceptions} on lui attribue l'\code{id} déjà donné à 
cette valeur.
\item dans le cas contraire,
elle est stockée dans un nouvel emplacement mémoire et reçoit un nouvel \code{id}
\end{itemize}
\end{itemize}

Le but de ceci est d'économiser la mémoire en faisant en sorte que si on a mille 
variables de même valeur 2, elles pointent toutes (autant que possible) vers un 
même case mémoire qui contient la valeur 2.
% Cette manœuvre a un coût, c'est pour ça que le choix de Python est de ne faire
% que partiellement l'interning. 

\medskip
Par contre, on insiste bien sur le fait que à chaque affectation, on crée une 
nouvelle variable : comme on le verra ci-dessous, si on écrit \code{a=2} puis 
\code{a=5}, les deux \code{a} sont des variables différentes en terme 
d'emplacement mémoire, de même si on écrit \code{a=a+1}.
\subsubsection*{Activité 2}
Dans la console Python, tapez les commandes suivantes et commentez les résultats :
\begin{Python}
a=5
b=a
c=3+2
d=int('2')
e=a+1
id(a)
id(b)
id(c)
id(d)
id(e)
\end{Python}
On retrouve bien le fait que la plupart des variables qui ont pour valeur 2 ont 
le même id.
\begin{Python}
a=8
b=a
id(a)
id(b)
a=50
b
id(a)
id(b)
\end{Python}

Les quatre premières lignes sont classiques. Puis, au moment de l'affectation 
\code{a=50}, on crée une nouvelle variable, à un nouvel emplacement mémoire. 
Et comme cette variable a également comme identifiant \code{a}, \code{a} et 
\code{b} pointent maintenant vers des emplacements mémoires différents et comme 
\code{b} pointe vers l'ancien emplacement mémoire, la variable \code{b} conserve 
sa valeur initiale.

\bigskip
D'autres exemples seront donnés avec des variables de types construits.

\subsection{Comparaison de variables}
Il existe deux commandes permettant de tester l'égalité de variables : \code{==} 
et \code{is}.

Plus précisément, \code{==} teste l'égalité des valeurs tandis que \code{is} 
permet de tester que l'égalité des emplacements mémoires, des \code{id}.

\subsubsection*{Activité 3}
Dans la console Python, tapez les commandes suivantes et commentez les résultats :
\begin{Python}
a=2
b=2.0
a==b
a is b
\end{Python}

Là encore, d'autres exemples seront donnés avec des variables de types construits.

\section{Types construits}
Dans cette section, on évoque les modèles théoriques de types construits et leurs 
implémentations en Python. 
\subsection{Tableaux}
\begin{Definition}{}
Un tableau est une collection ordonnée de taille fixée d'éléments de même type, 
auxquels on peut accéder (pour lecture ou écriture) au moyen d'un indice.  
\end{Definition}

\medskip
Historiquement, cette définition se comprend car elle permettait
une gestion de mémoire et un accès efficace.
En effet, on déclarait un tableau de taille fixée dont tous les éléments étaient 
du même type et le programme réservait des emplacements mémoires consécutifs 
et de même taille et dès lors, un calcul très simple permettait de trouver 
l'emplacement mémoire correspondant à chaque indice du tableau.

\medskip
Cependant, en Python, il n'y a pas de type \code{array} natif (même si on en 
trouve dans des modules comme \code{numpy}) et ce qui s'en rapproche le plus à 
l'usage est le type \code{list}  
qui est beaucoup plus souple d'usage car ses éléments peuvent être de différents 
types et on peut y ajouter des éléments.

\medskip
En Python, les éléments de type \code{list} sont représentés entre crochets 
\code{[ ]} et sont indexés à partir de $0$ (ces propriétés étant communes à de 
nombreux langages).

\subsubsection*{Activité 4}
Dans la console Python, tapez les commandes suivantes et commentez les résultats :
\begin{Python}
t=[2,5,7,-6,4]
type(t)
len(t)
t[0]
t[1]
t.append(10)
t
t.extend([7,8,9])
t
\end{Python}
On peut aussi illustrer sur ce type les règles d'affectation discutées précédemment. 
\begin{Python}
t=[1,2,3]
u=t
v=[1,2,3]
id(t)
id(u)
id(v)
t==v
t is v
t==u
t is u
t[0]=5
u.append(10)
t
u
\end{Python}
On observe bien que \code{u} et \code{t} mènent à un même emplacement 
mémoire. Et donc, toute modification de l'une, que ce soit au moyen d'un 
\code{.append} ou en modifiant une composante, modifie l'autre (vu qu'en 
fait ils ne font qu'un).

Par contre, si on refait une affectation \code{t=[...]}, alors on crée une nouvelle 
variable \code{t} avec un nouvel \code{id} et qui est donc indépendante de \code{u}.
\subsubsection*{Problèmes de copies}
Copier une liste de la manière précédente conduit très souvent à des problèmes
inattendus à l'exécution. Il convient de dupliquer la liste de diverses manières :    
\begin{Python}
a=[7,12,-11,2,0] 
b=list(a)
c=a[:]
d=a+[]
\end{Python}
On vérifie facilement que \code{a}, \code{b}, \code{c} et \code{d} sont des 
listes contenant les mêmes éléments mais non liées.

Cependant ce genre de duplication ne créé de double que de la liste, pas de ses 
éléments (c'est un clonage simple : un seul niveau). Comme on peut le voir dans
la suite :
\begin{Python}
a=[1,2,3]
b=[4,5,6] 
c=[a,b] 
d=list(c) 
\end{Python}
\code{c} et \code{d} ne sont pas liés : on peut rajouter un élément à \code{c} 
sans que cela impacte \code{d} (ou on peut vérifier qu'elles ont des \code{id} 
différents.): 
\begin{Python}
c.append([7,8,9]) 
c
d
\end{Python}
Mais \code{c[0]} et \code{d[0]}, qui sont deux alias de \code{a}, mènent à un 
seul et même emplacement mémoire !! 
\begin{Python}
c[0][1]=13 
c 
d 
\end{Python}
Pour parer à ce genre de difficultés, on peut utiliser des bibliothèques 
spécialement conçues : 
\begin{Python}
import copy  # importe la bibliothèque copy
d=copy.deepcopy(c) # crée une copie de c complètement indépendante 
\end{Python}
Ici, le clonage est fait en profondeur: on clone la liste, ses éléments, les 
éléments dans les éléments\dots
\subsection{Les p-uplets}
\begin{Definition}{}
Un p-uplet est une collection ordonnée de taille fixée d'éléments auxquels on 
peut accéder pour lecture au moyen d'un indice.

Un p-uplet est une structure immuable :
	\begin{itemize}
	\item on ne peut modifier ses éléments.
	\item on ne peut lui rajouter des éléments
	\end{itemize}
\end{Definition}

\medskip
En Python, les p-uplets (\code{tuple}) sont présentés par des suites d'éléments
séparés par des virgules, éventuellement encadrés par des parenthèses (non 
obligatoires sauf pour le tuple vide).

\medskip
\begin{Python}
tuple1=(2,5,7,'abc',3.25)
tuple1[2]
tuple1[0]
len(tuple1)
type(tuple1)
tuple2=5,9,'bonjour'
type(tuple2) # les parenthèses sont inutiles
tuple3=()
type(tuple3)
tuple4=(3)
type(tuple4) # ATTENTION !! (3) est un int
tuple4=(3,)
type(tuple4) # Là, c est mieux !!
\end{Python}

Notons que lorsqu'on réalise une affectation simultanée ou lorsqu'on considère 
une fonction qui renvoie plusieurs valeurs, on utilise la structure de tuple.

\subsubsection*{Caractère immuable}
On voit maintenant la différence essentielle : l'impossibilité de faire des 
affectations sur des tuples. Testons les commandes suivantes en prenant bonne 
note des messages d'erreurs obtenus. 
\begin{Python}
tuple1=(2,5.8,'abc',14)
tuple1.append(7)
tuple1[3]=78
\end{Python}

\medskip
\includegraphics[width=14cm]{images/erreurtuple1.JPG}

\vspace{1cm}
\includegraphics[width=14cm]{images/erreurtuple2.JPG}

\subsection{Les Dictionnaires}
\subsubsection*{Premières définitions} 
Les tableaux sont un bon outil quand on a des données de même nature et qui apparaissent 
dans un ordre naturel : 1$^{ere}$ note obtenue, 2$^{eme}$ note obtenue,.... 

Mais si on veut regrouper dans une même structure des données de type différents 
et pour lesquels il n'y a pas de raisons d'avoir un ordre particulier, alors la 
structure de dictionnaire est la plus adaptée. 
\begin{Definition}{} Un dictionnaire est une structure de données dans lequel 
les éléments sont représentés par des couples clés-valeurs, les clés servant pour 
accéder aux valeurs.

Un dictionnaire est une structure mutable :
	\begin{itemize}
	\item on peut modifier les valeurs
	\item on peut lui rajouter des couples clés-valeurs
	\end{itemize}
\end{Definition}

En Python, les dictionnaires sont écrits entre accolades, chaque couple 
\code{clé:valeur} étant séparé par une virgule. 

Par exemple,
\begin{Python}
eleve1={"nom":"Dupont","prenom":"Nadia","sexe":0,"anglais":True,"notesAnglais":[15,17,9,12],"allemand":False,"notesAllemand":[]}
eleve1["nom"]
eleve1["anglais"]
eleve1["notesAnglais"].append(18)
eleve1["nom"]="Dupond"
\end{Python}

$\bullet$ Ici, les clefs sont :

\code{"nom","prenom","sexe","anglais","notesAnglais","allemand","notesAllemand"}

On peut y accéder directement au moyen de :
\begin{Python}
eleve1.keys()
\end{Python}

$\bullet$ Ici, les valeurs sont :

\code{"Dupont","Nadia",0,True,[15,17,9,12],False,[]}

On peut y accéder directement au moyen de :
\begin{Python}
eleve1.values()
\end{Python}

$\bullet$ On peut accéder aux couples \code{clé:valeur} (représentés comme des 
2-uplets) au moyen de \code{items} : 
\begin{Python}
eleve1.items()
\end{Python}
\subsubsection*{Nature des clefs}
Dans l'exemple ci-dessus, il est naturel d'avoir pour clef des chaînes de caractères. \\
Mais en fait, les clefs peuvent être des \code{int}, \code{float} ou même des tuples.
\begin{Python}
essai={1:3,1.5:7,2.8:9,(1,2):10}
essai[1]         # renvoie 3
essai[2.8]       # renvoie 9
essai[(1,2)]     # renvoie 10
\end{Python}
\subsubsection*{Ajout d'une clef}
Contrairement à ce qui se passe avec un type \code{list}, on peut ajouter "à la volée" un 
couple clef:valeur à un dictionnaire.

\begin{Python}
eleve1["regime"]="externe"
eleve1
\end{Python}
Le couple \code{"regime":"externe"} a été ajouté au dictionnaire \code{eleve1} 

\subsubsection*{Test sur les éléments d'un dictionnaire}

Les divers tests s'effectuent au moyen du mot-clef  \code{in}.

Les tests suivants existent :  
\begin{Python}
clef in dictionnaire
# teste si une clef est dans le dictionnaire
clef in dictionnaire.keys()
# teste si une clef est dans la liste des clefs
(clef,valeur) in dictionnaire.items()
# teste si un couple clef:valeur est dans la liste des couples du dictionnaire
valeur in dictionnaire.values()
# teste si une valeur est dans la liste des valeurs
\end{Python}
\`A noter que le test le moins coûteux informatiquement est \code{clef in dictionnaire}, 
les autres tests obligeant d'abord à créer les listes des clefs/valeurs/objets.

Le test \code{value in dictionnaire}, quant à lui, ne fonctionne pas. 

Par exemple,
\begin{Python}
"prenom" in eleve1                   # renvoie True
"Nadia" in eleve1                    # renvoie False
"prenom" in eleve1.keys()            # renvoie True
("prenom","Nadia") in eleve1.items() # renvoie True
"Nadia" in eleve1.values()           # renvoie True
\end{Python}
\subsubsection*{Parcours d'un dictionnaire}
On peut déduire du comportement des tests que l'on peut faire un parcours sur 
les clés d'un dictionnaire mais pas sur ses valeurs.

Ainsi, quand on écrit \code{for k in dictionnaire}, la variable \code{k} va 
parcourir l'ensemble des clefs du dictionnaire. 
Par exemple, les commandes 
\begin{Python}
for k in eleve1:
	print(k)
\end{Python}
provoquent l'affichage de :

nom

prenom

sexe

anglais

... 
\medskip\\
On peut aussi tester de manière similaire : 
\begin{Python}
for k in eleve1:
	print(k,eleve1[k])
\end{Python}
ou encore 
\begin{Python}
for k in eleve1.items():
	print(k)
\end{Python}
ou encore 
\begin{Python}
for k in eleve1.items():
	print( k[0]+" a la valeur "+ str(k[1]) )
\end{Python}

\subsubsection*{Activité 5}
On crée la variable 
\begin{Python}
D={'400' : 'chute de Rame', '1492' : 'decouverte Amerique', '1789' : 'prise Bastille','1945' : 'fin 2eme GM'}
\end{Python}
\begin{enumerate}
\item Quel est le type de \code{D} ?
\item Que renvoie \code{len(D)} ?  
\item Que renvoie \code{D[0]} ?
\item Donner les clefs de \code{D}. Quelle commande permet d'y accéder ?
\item Citer une valeur de \code{D}. Comment obtenir la liste des valeurs ?
\item Proposer une commande pour corriger la faute d'orthographe \code{ 'Rame'} au lieu de \code{'Rome'}.
\item Proposer une commande pour ajouter une entrée concernant la fin de la première guerre mondiale à \code{D}. 
\end{enumerate} 
\end{document}
