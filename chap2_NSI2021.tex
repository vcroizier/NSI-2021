\documentclass[11pt,a4paper,french,twoside]{PMCours}
\usepackage{hyperref}
% TODO: Ajuster la position des sauts de page.
\begin{document}
\TitreISN{Classe de Terminale}{Année 2021--2022}
{Numérique et Sciences Informatiques}{Chap. 2 : Quelques notions sur la récursivité}
\section{Définition}
\begin{Definition}{}
\begin{enumerate}
\item Un algorithme récursif est un algorithme qui résout un problème 
en calculant des solutions d'instances plus petites du même problème.
\item Une fonction récursive est une fonction qui s'appelle elle-même 
dans sa définition. 
\end{enumerate}
\end{Definition}

Par exemple, écrivons une fonction permettant de calculer la somme 
des $n$ premiers entiers non nuls. 

Sous forme classique,
\begin{Python}
def somme(n):
	S=0
	for k in range(1,n+1):
		S=S+k
	return S
\end{Python}
En version récursive, on obtient le programme 
\begin{Python}
def somme_rec(n):
	if n==1:
		return 1
	else :
		return n+somme_rec(n-1)
\end{Python}
Détaillons le fonctionnement de \code{somme\_rec} pour $n=4$ en détaillant 
la pile d'exécution (\emph{call stack}) relative à ce programme.
%à ce moment, je leur fais à la main le cheminement du programme

\newpage
On observe qu'une fonction récursive s'écrit avec : 
\begin{itemize}
\item un (ou plusieurs) cas terminal pour lequel la fonction renvoie 
directement une valeur;
\item un cas général pour lequel la fonction fait appel à elle-même. 
\end{itemize}

Un exemple dans lequel on a deux cas terminaux, la célébrissime 
suite de Fibonacci, définie par 
$\sys{
F_0=0\\
F_1=1\\
F_{n+2}=F_{n+1}+F_n\text{ si $n\geq 0$}
}$


Elle se programme par récursivité au moyen de 

\begin{Python}
def Fib(n):
	if n==0:
		return 0
	elif n==1:
		return 1
	else :
		return Fib(n-1)+Fib(n-2)
\end{Python}

Calculons à la main les premiers termes et observons les appels occasionnés
par l'exécution de \code{Fib(5)}.
\vspace{7cm}

\section{Terminaison et coût d'exécution d'une fonction récursive}
\subsection{Terminaison} 
Si on veut être certain de la terminaison d'un appel d'une fonction récursive,
on doit vérifier que quelle que soit la valeur de départ des arguments, 
toutes les piles d'appels engendrées par l'exécution du programme se termineront
en un nombre fini d'étapes par des cas terminaux.

Ceci reprend le principe des {\em variants de boucle} utilisés pour prouver 
la terminaison d'une boucle \code{while}.

Par exemple, en reprenant \code{somme\_rec}. Soit \code{n} un entier $\geq1$.
\begin{itemize}
\item si \code{n} vaut 1, l'exécution de \code{somme\_rec(n)} s'arrête , 
on a bien la terminaison.
\item si \code{n}>1, on appelle \code{somme\_rec(n-1)} et donc le problème 
de terminaison pour \code{somme\_rec(n)} se ramène exactement à celui 
de \code{somme\_rec(n-1)} 
\end{itemize}
Par suite, si on commence avec un entier \code{n}$\geq1$,
on sait que l'exécution du programme s'arrêtera en  \code{n-1} étapes.

Par contre, si on essaye de calculer \code{somme\_rec(1/2)}, on n'arrive 
jamais à la valeur terminale et on fait donc une infinité d'appels récursifs. 

Ceci provoque une erreur spécifique, le dépassement de la taille de la pile 
d'appel, un \emph{stack overflow}.
 
\includegraphics[width=14cm]{images/erreurrec.JPG}    

Il vaut mieux donc commencer une fonction récursive par l'utilisation d'une commande
\code{assert} pour éviter tout mauvais choix d'arguments.

\subsection{Coût d'exécution} 
Il peut être assez difficile à calculer, dépendant du nombre d'appels récursifs
faits et du coût d'exécution de chaque appel. 


Par exemple, pour \code{somme\_rec}, on a un test et une addition à chaque appel
de la fonction, ce qui fait de l'ordre de 2n opérations pour exécuter 
\code{somme\_rec(n)}. 


Il faut noter que le coût d'exécution de fonctions récursives peut être 
extrêmement élevés.


Par exemple, si on s'intéresse au nombre d'appels causés par l'exécution de
\code{Fib(n)}, on voit qu'il est de l'ordre de \code{Fib(n)} car il obéit 
au même principe : on additionne le nombre d'appels occasionnés par l'exécution 
de \code{Fib(n-1)} et de \code{Fib(n-2)} pour avoir le nombre d'appels 
occasionnés par l'exécution de \code{Fib(n)}.


On peut être plus précis grâce au programme suivant, compter 
le nombre exact d'appels :

\begin{Python}
nap=0

def Fib(n):
    global nap
    nap=nap+1
    if n==0:
        return 0
    elif n==1:
        return 1
    else :
        return Fib(n-1)+Fib(n-2)
\end{Python}

On constate que pour calculer \code{Fib(20)},
on doit faire 21891 appels de fonctions !! 

Ici, une programmation classique serait beaucoup moins coûteuse
(quelques dizaines d'opérations).

À noter que l'optimisation de tels programmes en mémorisant les résultats
des appels se nomme la \emph{mémoïsation}, et qu'elle relève du domaine 
de la \emph{programmation dynamique}.
 
\end{document}
