\documentclass[11pt,a4paper,french,twoside]{PMCours}
\usepackage{hyperref}
% TODO: Ajuster la position des sauts de page.
\begin{document}
\TitreISN{Classe de Terminale}{Année 2020--2021}
{Numérique et Sciences Informatiques}{Chap. 3 : Quelques notions sur la 
Programmation Orientée Objet}
\section{Définition}
\begin{Definition}{}
\begin{enumerate}
\item En informatique, une {\color{red} classe} est une structure de données dont 
les composantes peuvent être de natures variées. 
\item Les composantes d'une classe s'appellent ses {\color{red} attributs} et 
sont référencées par des noms.
\item La structure d'une classe peut également contenir des fonctions, qui 
s'appellent ses {\color{red} méthodes} et dont le but premier est la manipulation 
des attributs de la classe.
\item On appelle {\color{red} objet} une instance d'une classe.
\end{enumerate}
\end{Definition}

\section{Exemple en langage Python}
On propose un exemple dont le code est donné page 3.

Remarquons les points suivants :
\begin{enumerate}
\item En Python, on définit une classe grace au mot-clef \code{class} et le nom 
de la classe (ici \code{Personnage}) doit commencer par une majuscule.
\item La fonction \code{\_\_init\_\_} s'appelle le {\bf constructeur de la classe} 
et c'est par son utilisation que l'on crée des objets cette classe.

On remarque l'utilisation du mot-clef \code{self} qui, à l'intérieur de la définition
de la classe, désigne l'objet sur lequel on travaille.

\item En regardant le constructeur de la classe, on constate que la classe 
\code{Personnage} possède sept attributs :
\begin{itemize}
    \item \code{nom} et \code{race} dont les valeurs sont de type \code{str};
    \item \code{age}, \code{force}, \code{vitalite}, \code{vitalite\_max} dont les 
    valeurs sont de type \code{int};
    \item \code{en\_vie} dont la valeur est de type \code{bool}.
\end{itemize}
On remarque que à la création d'un objet, les valeurs des attributs peuvent être fixées 
par l'utilisateur ou peuvent être déterminées (constantes, tirées au hasard, etc...) 
dans le constructeur.
\item \code{\_\_str\_\_}, \code{affiche\_carac}, \code{perd\_vie}, 
\code{mange\_haricot}, \code{attaque} sont des méthodes de la classe.

Elles prennent toutes au moins un argument qui est \code{self} (qui référence 
l'objet lui-même).

Elles peuvent avoir d'autres arguments, de tous types, y compris objet de cette 
classe ou d'une autre.

\end{enumerate}

\subsection*{Activite}
\begin{enumerate}
\item Les deux commandes suivantes créent les variables \code{goku} et \code{krilin},
qui sont des instances de la classe \code{Personnage}.
\begin{Python}
goku=Personnage("San Goku","Saiyan",32)
krilin=Personnage("Krilin","Humain",35)
type(goku)
type(krilin)
\end{Python}
    \item On peut accéder aux attributs avec la syntaxe \code{objet.attribut} :
\begin{Python}
goku.nom
goku.vitalite
krilin.force
\end{Python} 
    On peut aussi les modifier "à la main" :
\begin{Python}
goku.nom="Kakarot"
krilin.force=500
\end{Python} 
    Mais même si cela est possible informatiquement, c'est considéré comme étant 
    de la mauvaise pratique en terme de programmation et normalement, les attributs 
    d'un objet ne doivent être modifiés que via des méthodes dédiées.

Ceci est lié au principe d'{\em encapsulation } : une classe est une structure 
de données fournie par un concepteur à un utilisateur ({\em client}).

Le client doit connaître les actions qu'il peut accomplir sur ces données. 
Ceci conduit à une énumération des fonctions, l'{\em interface} de la classe. 

Mais le client ignore a priori l'{\em implémentation} de la classe et modifier 
des attributs à la volée peut avoir des {\em effets de bords} non désirés. 

\item Les méthodes dont le nom commencent par une lettre s'utilisent avec la syntaxe


\code{objet.methode(arguments éventuels)}
\begin{Python}
goku.affiche_carac()
krilin.affiche_carac()
krilin.attaque(goku)
goku.attaque(krilin)
krilin.perd_vie(15)
\end{Python}

\item Les méthodes dont le nom est encadré par des \code{\_\_} correspondent à 
des opérations prédéfinies de Python et on les appelle avec des mots-clefs ou des
symboles plutôt qu'avec la syntaxe \code{objet.methode}

Par exemple, \code{\_\_str\_\_} est appelée par le mot clef \code{str} et renvoie 
une chaîne de caractères décrivant l'objet.

\begin{Python}
str(goku)
\end{Python}

De même,
\begin{itemize} 
\item \code{\_\_lt\_\_} permet de définir une méthode qui effectue des tests, 
renvoie un booléen et est appelée par \code{<}.
\item \code{\_\_eq\_\_} permet de définir une méthode qui effectue des tests, 
renvoie un booléen et est appelée par \code{==}. En effet, par défaut, le \code{==}  
appliqué à des objets réagira comme le \code{is} et ne renverra \code{True} que si 
les deux identifiants comparés renvoient au même \code{id}.

Par exemple, le code :
\begin{Python}
    def __eq__(self,autre):
        return(self.force+self.vitalite==autre.force+autre.vitalite)
\end{Python}

définit une méthode et le test \code{goku==krilin} renverra \code{True} si 
on obtient le même total en additionnant leurs \code{force} et \code{vitalite}. 
\end{itemize}
\end{enumerate}


\subsection*{Code Python de la classe}
\begin{Python}
import random

class Personnage:
    def __init__(self,lenom,larace,lage):
        self.nom=lenom
        self.race=larace
        self.age=lage
        self.en_vie=True
        if self.race=="Saiyan":
            self.force=random.randint(30,40)
            self.vitalite=100
            self.vitalite_max=100
        elif self.race=="Humain":
            self.force=random.randint(10,20)
            self.vitalite=30
            self.vitalite_max=30
        
    def __str__(self):
            return ("Bonjour, je m'appelle "+self.nom+" de la race des "+self.race+"s\n. J'ai "+str(self.force)+" ans.")
        
    def affiche_carac(self):
            print(str([self.nom,self.force,self.vitalite]))  

    def perd_vie(self,n):
        self.vitalite=self.vitalite-n
        print(self.nom+" a encore une vitalité de "+str(self.vitalite))
        if self.vitalite<=0:
            self.en_vie=False
            print(self.nom+" est mort.")
        
    def mange_haricot(self):
        self.vitalite=self.vitalite_max
        
    def attaque(self,autre):
        autre.perd_vie(self.force)
\end{Python}


\end{document}
